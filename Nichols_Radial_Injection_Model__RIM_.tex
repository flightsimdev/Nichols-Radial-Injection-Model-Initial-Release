\documentclass[11pt]{article}

\usepackage[a4paper,margin=1in]{geometry}
\usepackage{amsmath,amssymb}
\usepackage{graphicx}
\usepackage{booktabs}
\usepackage{hyperref}
\usepackage{physics}

\title{%
Nichols Radial Injection Model (RIM) and Radial ERB Inflows:\\
A Mechanism for Progenitor-less Astrophysical Events,\\
the Hubble Pulse, and 4D Substrate Interactions
}

\author{Lawrence William Nichols}
\date{January 21, 2026}

\begin{document}
\maketitle

\begin{abstract}
Current $\Lambda$CDM models struggle to account for the rapid formation of massive
galaxies and hours-long Gamma-Ray Bursts (GRBs) observed by JWST. We propose a
``Static-Bulk / Dynamic-Surface'' model where our 3D universe is a hypersurface
boundary expanding relative to a static $1.46$ trillion light-year diameter 4D
manifold. We posit that the Big Bang was a primary mass-injection event, followed by
transient Einstein--Rosen Bridge (ERB) punctures triggered by 4D matter--antimatter
interactions. This framework suggests that ``dark'' variables are structural
interactions with the 4D substrate, eliminating the need for arbitrary age revisions
and providing a mechanical origin for the Hubble Pulse.
\end{abstract}

\section*{Research Provenance}
The Radial Injection Model (RIM) is the culmination of a six-month iterative process.
Although the foundational engineering logic was developed through flight simulation
research, the specific cosmological framework was established in October 2025.

\section{The Nichols Thought Experiments Timeline}

\begin{itemize}
\item \textbf{2013--2025}: Development of mechanical and fluid-dynamic logic via the
\textit{flightsimdev} research platform.
\item \textbf{August 20, 2025}: The ``Science Pivot''---publication of the first
science-focused video applying engineering principles to anomalies in the
$\Lambda$CDM model.
\item \textbf{October 2025}: Formal initiation of \textit{The Nichols Thought
Experiments} (TE 1--18), defining the 4D manifold and ``Guest Space'' hypothesis
[cite: 31].
\item \textbf{January 2026}: Identification of the ``Hubble Pulse'' and verification
via progenitor-less events such as GRB~250702B [cite: 36, 40].
\end{itemize}

\subsection{The Video 18 Mass-Parity Discovery}

Empirical testing suggests a near-perfect parity between daily 4D injection and 3D
sequestration. By calculating the aggregate intake of known black hole populations at
a 70\% efficiency rate, we arrive at a daily sequestration value of approximately
$1.5 \times 10^{53}\,\mathrm{kg}$. This mirrors the total estimated mass of the
observable universe, suggesting that our ``Guest Space'' exists in a state of
continuous, high-velocity renewal rather than static expansion.

\section{In-Situ Stellar Augmentation}

We propose that ERB events may occur within existing stellar cores, where radial mass
flux $\Phi_m$ acts as a secondary fuel source. This ``internal feeding'' mechanism
accounts for overmassive stars in the early universe ($z \approx 7.3$) that appear to
violate standard Eddington luminosity limits. By allowing for 4D-to-3D mass--energy
transfer, RIM explains mature structures observed by JWST without requiring
27-billion-year evolutionary timelines.

\section{Mechanics of Cosmic Expansion}

\subsection{ERB-Driven Volumetric Growth}

Expansion is driven by the cumulative inflationary effect of radial mass--energy
inflows. Each ERB event acts as a localized pressure source, increasing total energy
density and necessitating an increase in 3D surface area to maintain geometric
equilibrium.

\subsection{Calculation of the Hypersphere Curvature Radius}

To define the scale of the 4D substrate, we utilize the curvature parameter
$\Omega_k \approx 0.004$ and the observable radius $r = 46.5$ billion light-years. In
a near-flat 3D hypersurface, the curvature radius $R$ is derived as
\begin{equation}
R = \frac{r}{\sqrt{\Omega_k}}.
\end{equation}

Applying observed values,
\[
\sqrt{0.004} \approx 0.063245,
\qquad
R \approx \frac{46.5 \times 10^9}{0.063245}
\approx 7.35 \times 10^{11}\ \text{ly}.
\]

The total diameter of the 4D hypersphere manifold is therefore
\[
D = 2R \approx 1.47\ \text{trillion light-years}.
\]

\subsection{The Hubble Pulse: Correlation with $\Phi_m$}

Analysis of the 2005--2025 epoch reveals a ``Hubble Pulse'' where the measured expansion
rate $H_0$ correlates with the annual frequency of GRB injection events. This suggests
that $H_0$ is a dynamic function of the integrated mass flux,
\begin{equation}
H_0(t) \propto \sum \int \Phi_m(t)\, dt .
\end{equation}

\begin{table}[h]
\centering
\begin{tabular}{lccc}
\toprule
Epoch & GRB Frequency (yr$^{-1}$) & Measured $H_0$ & RIM Interpretation \\
\midrule
2011--2015 & $\sim 90$ & 69.0--71.0 &
Stabilization phase: consistent mass flux \\
2016--2020 & 100--105 & 73.2--75.8 &
Expansion surge via frequent 4D punctures \\
2021--2024 & $\sim 80$ & 67.4--70.4 &
Pressure drop from reduced injection \\
\bottomrule
\end{tabular}
\caption{Empirical correlation between annual GRB frequency and $H_0$ fluctuations.}
\end{table}

\section{The Pressure-Gradient Mechanism: Return Valves}

Inside a black hole, gravitational pressure $P_{\mathrm{BH}}$ exceeds bulk pressure
$P_B$, forcing a reverse-flow state
\[
\Phi_{\mathrm{reverse}} \propto (P_{\mathrm{BH}} - P_B).
\]
This drain remains undetected by current electromagnetic instruments because light is
the sequestered medium. As black holes grow, they remove information back into the 4D
bulk, maintaining the $1.46$ trillion light-year hypersphere's geometric equilibrium
and resolving the Hawking information paradox.

\section{The Substrate Hypothesis: Space as a Guest Structure}

We propose that the space we inhabit is a secondary 3D structure displacing a
pre-existing 4D manifold. This ``Guest Space'' logic suggests the space we sit on is
not intrinsically ours.

\begin{itemize}
\item \textbf{Reassigning fudge factors}: Dark Energy and Dark Matter are
recontextualized as surface tension and displacement signatures of the 4D substrate.
\item \textbf{Ancient wanderers}: Galaxies mature at $z > 10$ are 4D residents from
other regions of the hypersphere that have drifted into our observable 6\% slice.
\end{itemize}

\section{Empirical Predictions and Observational Signatures}

The Radial Injection Model (RIM) provides four primary testable predictions:

\begin{enumerate}
\item \textbf{Void-injection events}: Detection of GRBs in local voids lacking
detectable gas-cloud progenitors [cite: 109].
\item \textbf{High-energy kinetic cargo}: Energy spikes in the 700~keV to 4.3~MeV
range, exceeding the 511~keV annihilation line, indicating a 50\% matter bias in the
4D-to-3D mass flux.
\item \textbf{Sustained flux duration}: Hours-long events (e.g.\ GRB~250702B)
represent the stability period of the 4D conduit, injecting up to 400{,}000
Earth-masses of matter per event [cite: 110].
\item \textbf{Expansion jitter}: A 20\% increase in progenitor-less GRBs in 2026 will
lead to a spike in $H_0 \ge 74$~km~s$^{-1}$~Mpc$^{-1}$ by 2027 [cite: 111].
\end{enumerate}

\section{Conclusion}

By treating black holes as 4D anchors and GRBs as volumetric engines, RIM replaces
abstract dark variables with structural engineering. Placing the universe on a
hypersphere, rather than forcing it to be the hypersphere, reconciles a century of
cosmology with modern JWST observations.

\section*{References}

Einstein, A., \& Rosen, N.\ (1935). \textit{The Particle Problem in the General Theory
of Relativity}. Physical Review, 48(1), 73.

Euclid Collaboration (2025). \textit{Euclid I: Overview of the Euclid Mission}.
Astronomy and Astrophysics, 697.

Gardner, J.\ P., et al.\ (2006/2023). \textit{The James Webb Space Telescope Mission}.
Space Science Reviews, 123, 485--606.

Gehrels, N., et al.\ (2004). \textit{The Swift Gamma-Ray Burst Mission}. The
Astrophysical Journal, 611(2).

Riess, A.\ G., et al.\ (2024). \textit{A Comprehensive Measurement of the Local Value
of the Hubble Constant with HST and JWST}. The Astrophysical Journal Letters.

Segal, I.\ E.\ (1972). \textit{Theoretical Foundations of the Chronometric Cosmology}.

\end{document}
