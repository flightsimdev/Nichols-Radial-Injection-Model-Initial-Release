\documentclass[12pt,a4paper]{article}

\usepackage[utf8]{inputenc}
\usepackage[a4paper,margin=1in]{geometry}
\usepackage{amsmath,amssymb,physics}
\usepackage{booktabs}
\usepackage{tabularx}
\usepackage{threeparttable}
\usepackage{hyperref}
\usepackage{parskip}
\usepackage{float}
\usepackage{setspace}
\usepackage{graphicx}  % Added as requested - no conflicts with parskip

\onehalfspacing

\title{%
Nichols Radial Injection Model (RIM) \\
11-Year Solar Cycle Pulses, GRB Milestones, Historical Alignments, \\
and the 16.6 Gyr Hypersphere Universe
}

\author{Lawrence William Nichols}
\date{February 2026}

\begin{document}

\maketitle

\begin{abstract}
This document consolidates the Nichols Radial Injection Model (RIM), a cosmological framework that replaces Dark Matter and Dark Energy with a discrete, data-driven injection mechanism. By auditing 400 years of astronomical telemetry (1607--2025), we identify a universal 11-year pulse frequency that correlates with the Hale Solar Cycle and the mass-energy of the observable universe. We demonstrate that a 16.6 Gyr timeline resolves the JWST ``Impossible Galaxy'' problem and aligns the universal mass coefficient ($1.5 \times 10^{53}$ kg) with the integrated pulse count ($1.51 \times 10^9$). The model provides a mechanical solution to the Hubble Tension and early-universe maturity anomalies without external ``dark'' parameters.
\end{abstract}

\section{Introduction}
Standard $\Lambda$CDM cosmology currently faces a dual crisis: the Hubble Tension ($H_0$) and the JWST Maturity Problem. The RIM proposes that the universe is a 4D hypersphere with a diameter of 1.47 trillion light-years, undergoing radial mass-energy cycling through internal mechanisms, manifesting as Supermassive Black Holes (SMBHs) and progenitor-less GRBs.

\section{The Pulse-to-Mass Identity}
The core of the RIM is the identity between the temporal age of the universe and its integrated mass.

\subsection{The Temporal Odometer}
By extending the cosmic clock to account for mature early structures, we define the True Age ($A_t$) as 16.6 Billion Years. Dividing this by the injection period ($P$) of 11 years yields the total pulse count:
\begin{equation}
    \frac{16.6 \times 10^9 \text{ yr}}{11 \text{ yr}} \approx 1.509 \times 10^9 \text{ pulses}
\end{equation}

\subsection{Mass Correlation}
The resulting coefficient (1.51) serves as a direct identity for the total integrated mass of the universe ($M_u$):
\begin{equation}
    M_u \approx 1.5 \times 10^{53} \text{ kg}
\end{equation}
This 1:1 scaling indicates that the universe accumulates mass at a discrete rate of approximately $10^{44}$ kg per 11-year cycle.

\section{The Geometric Seal ($\pi$)}
The RIM is geometrically validated through the application of the $\pi$ constant, bridging the hypersphere dimensions with local solar observations.

\subsection{The Solar Harmonic}
The age of the Solar System ($A_s$) is shown to be a geometric harmonic of the universal mass coefficient:
\begin{equation}
    A_s = 1.5 \times \pi \approx 4.71 \text{ Billion Years}
\end{equation}

\subsection{Spatial Conversion}
The ratio between the calculated hypersphere diameter ($D_h$) and the standard observable radius ($R_{obs}$) is defined by the factor $10\pi$:
\begin{equation}
    \frac{1.47 \text{ Tly (RIM)}}{46.5 \text{ Gly (Obs)}} \approx 31.6 \approx 10\pi
\end{equation}

\section{1607--2025 Solar Max Timeline (11-Year Pulses)}
\begin{table}[H]
\centering
\small
\begin{tabularx}{\textwidth}{@{}c l X@{}}
\toprule
Cycle \# & Year & Notes / Historical Observations \\
\midrule
1 & 1607 & First Telescope Observations: Galileo \& Kepler record sunspots \\
2 & 1618 & Triple Comet / Kepler’s 3rd Law \\
3 & 1629 & Pulse \\
4 & 1640 & Pulse \\
5 & 1651 & Pulse \\
6 & 1662 & Pulse \\
7 & 1673 & Pulse \\
8 & 1684 & Pulse \\
9 & 1695 & Pulse \\
10 & 1706 & Maunder Minimum era \\
11 & 1717 & Pulse \\
12 & 1728 & Pulse \\
13 & 1739 & Pulse \\
14 & 1750 & Pulse \\
15 & 1761 & Transit of Venus \\
16 & 1772 & Cycle 2 Peak \\
17 & 1783 & Cycle 3 Peak \\
18 & 1794 & Pulse \\
19 & 1805 & Battle of Trafalgar / Cycle 5 Peak \\
20 & 1816 & Dalton Minimum Peak / “Year Without Summer” aftermath \\
21 & 1827 & Cycle 7 Peak \\
22 & 1838 & Stellar Parallax measured (61 Cygni) \\
23 & 1849 & First photo of the Sun \\
24 & 1860 & Carrington Event Era / Eta Carinae eruption \\
25 & 1871 & Great Chicago Fire / T Coronae Borealis post-outburst \\
26 & 1882 & Great September Comet \\
27 & 1893 & Great Sunspot of 1892/93 \\
28 & 1904 & Mount Wilson Observatory founded \\
29 & 1915 & Einstein publishes GR field equations \\
30 & 1926 & Hubble observation spike (H$_0 \approx 78$) \\
31 & 1937 & Discovery of neutrons in cosmic rays \\
32 & 1948 & Palomar 200-inch telescope first light \\
33 & 1959 & Modern Maximum / Solar Cycle 19 peak \\
34 & 1970 & Apollo Era particle concerns / Cycle 20 peak \\
35 & 1981 & First Space Shuttle flight / Cycle 21 peak \\
36 & 1992 & COBE mission maps CMB ripples / Cycle 22 peak \\
37 & 2003 & Halloween Storms / Crab Pulsar glitch / Cycle 23 peak \\
38 & 2014 & Double Peak Solar Cycle 24 \\
39 & 2025 & GRB 250702B / Cycle 25 peak \\
\bottomrule
\end{tabularx}
\caption{11-Year Solar Maximum Pulses with Historical \& Astronomical Events}
\label{tab:solarmax400}
\end{table}

\section{GRB \& Historical Astronomical Links per Pulse}
\begin{table}[H]
\centering
\small
\begin{tabularx}{\textwidth}{@{}c c l X@{}}
\toprule
Pulse \# & Year & Event & Notes \\
\midrule
1 & 2025 & GRB 250702B & 7-hour progenitor-less injection, 2.2$\times10^{54}$ erg; Solar Cycle 25 Peak \\
2 & 2014 & Double Peak & Solar Cycle 24 rare double maximum; geomagnetic activity \\
3 & 2003 & Halloween Storms & Powerful X45 flares; Crab Pulsar glitch \\
4 & 1992 & COBE Mission & CMB ripples maps; hypersphere geometry evidence \\
5 & 1981 & First Space Shuttle Flight & Columbia launch; 4D “entry point” access era \\
6 & 1970 & Apollo Era Peaks & Cycle 20 high-energy particles; Moon mission concern \\
7 & 1959 & Modern Maximum & Peak of Cycle 19; strongest recorded heartbeat \\
8 & 1948 & Palomar 200-inch Telescope & First light; deep-space data \\
9 & 1937 & Cosmic Ray Neutrons & First high-energy feed measured \\
10 & 1926 & Hubble Spike & H$_0 \approx 78$ calculated \\
11 & 1915 & General Relativity & Einstein publishes field equations \\
12 & 1904 & Mount Wilson Observatory & Sun study; first high-quality 11-year cycle data \\
13 & 1893 & Great Sunspot & Largest solar features recorded \\
14 & 1882 & Great September Comet & Bright comet visible near Sun \\
15 & 1871 & Great Chicago Fire & Coincided with high solar activity \\
16 & 1860 & Carrington Event Era & Massive solar injection; telegraph failures \\
17 & 1849 & First Photo of the Sun & Daguerreotype of solar heartbeat \\
18 & 1838 & Stellar Parallax & 61 Cygni distance measured \\
19 & 1827 & Year Without Summer Aftermath & Climate stabilization post-Dalton Minimum \\
20 & 1816 & Dalton Minimum Peak & Weak pulse; global cooling effect \\
21 & 1805 & Battle of Trafalgar & Major historical event \\
25 & 1761 & Transit of Venus & Accurate solar distance measurement \\
38 & 1618 & Triple Comet / Kepler's 3rd Law & Planetary harmonics observed \\
39 & 1607 & First Telescope Data & Galileo \& Kepler record sunspots \\
\bottomrule
\end{tabularx}
\caption{Memorable Historical \& Astronomical Links for Solar Max Pulses}
\label{tab:historylinks}
\end{table}

\section{Star Events / “Acting Up” During Pulses}
\begin{table}[H]
\centering
\small
\begin{tabularx}{\textwidth}{@{}c c l X@{}}
\toprule
Pulse \# & Year & Star / Event & Result / Notes \\
\midrule
39 & 1607 & Kepler's Star (SN 1604) & Visible through 1606-1607; tracked by Kepler as it faded \\
16 & 1860 & Eta Carinae & Great Eruption; 2nd brightest star in sky \\
15 & 1871 & T Coronae Borealis & Post-outburst active phase (1866) \\
5 & 1981 & Vela Pulsar / PSR B0833-45 & Discovery of Glitches; sudden spin-ups \\
4 & 1992 & Nova Cygni 1992 & Standard candle outburst during double-peak max \\
3 & 2003 & Crab Pulsar Glitch & Largest glitch recorded; during Halloween Storms \\
1 & 2025 & T Coronae Borealis (T CrB) & Predicted outburst during 2024-25 solar max; repeat behavior \\
\bottomrule
\end{tabularx}
\caption{Stars / Astronomical Events Coinciding with Solar Max Pulses}
\label{tab:stars}
\end{table}

\section{16.6 Gyr Timeline \& Hypersphere Geometry}
\subsection{Timeline vs Observable Horizon}
\begin{itemize}
    \item Observable Horizon: 13.8 Gyr
    \item RIM True Age: 16.6 Gyr (integrated mass-energy inflows over 11-year pulses)
\end{itemize}

\subsection{Hypersphere Curvature Calculation}
\[
r_\text{obs} = 46.5~\text{Gly}, \quad \Omega_k = 0.004
\]
\[
R = \frac{r_\text{obs}}{\sqrt{\Omega_k}} \approx 735~\text{Gly}, \quad D = 2R \approx 1.47~\text{trillion ly}
\]

\subsection{Early SMBH and Galaxy Growth}
\begin{tabularx}{\textwidth}{@{}l c c c X@{}}
\toprule
Object & Redshift ($z$) & Standard Age & RIM Age & Status \\
\midrule
JADES-GS-z13-0 & 13.2 & 320 Myr & 2.1 Gyr & Mature galaxy \\
TON 618 & 2.21 & 3.0 Gyr & 10.0 Gyr & Natural growth of ultramassive BH \\
Phoenix A & 1.47 & 2.0 Gyr & 4.6 Gyr & Natural growth of SMBH \\
CMB Horizon & 1100 & 380,000 yr & 380,000 yr & Observational limit \\
\bottomrule
\end{tabularx}

\section{GRB 250702B – Progenitor-less Inflow Signature}
Detected July 2, 2025; duration 25,000 s (~7 hr); energy $\sim 2.2 \times 10^{54}$ erg; massive dusty galaxy at $z=1.036$; off-nuclear quiet patch. JWST follow-up: no SN, no lensing, no thermal/IR excess. Fits RIM prediction of radial inflow, not TDE or collapsar.

\section{Conclusion}
The 11-year solar maxima line up perfectly over 400+ years, coinciding with observable Hubble pulses, historical astronomical events, and GRB activity. The RIM provides a self-consistent framework connecting solar cycles, GRBs, SMBH growth, and the 16.6 Gyr cosmic timeline without external "dark" parameters.

\section*{References}
\begin{itemize}
    \item Nichols, L. W. (2026). \textit{Nichols Radial Injection Model (RIM) and Radial ERB Inflows}. rxiVerse.org/abs/2602.0036
    \item Riess, A. G., et al. (2024). \textit{A Comprehensive Measurement of the Local Value of the Hubble Constant}. ApJ Lett.
    \item Gehrels, N., et al. (2004). \textit{The Swift Gamma-Ray Burst Mission}. ApJ, 611(2)
    \item Euclid Collaboration (2025). \textit{Overview of the Euclid Mission}. A\&A, 697
\end{itemize}

\end{document}