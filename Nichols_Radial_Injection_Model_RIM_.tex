\documentclass[12pt]{article} % Increased font to 12pt for readability

\usepackage[a4paper,margin=1.25in]{geometry} % Slightly wider margins
\usepackage{amsmath,amssymb}
\usepackage{graphicx}
\usepackage{booktabs}
\usepackage{hyperref}
\usepackage{physics}
\usepackage{tabularx}
\usepackage{threeparttable}
\usepackage{setspace} % Package for line spacing
\onehalfspacing % Set line spacing to 1.5

\usepackage{parskip} % Automatically adds space between paragraphs and removes indent
\setlength{\parskip}{1em} % Increases that space to a full line height

\title{%
Nichols Radial Injection Model (RIM) and Radial ERB Inflows:\\
A Mechanism for Progenitor-less Astrophysical Events,\\
the Hubble Pulse, and 4D Substrate Interactions
}

\author{Lawrence William Nichols}
\date{January 21, 2026}

\begin{document}
\maketitle

\begin{abstract}
Current $\Lambda$CDM models struggle to account for the rapid formation of massive galaxies [cite: 5] and hours-long Gamma-Ray Bursts (GRBs) observed by JWST[cite: 5]. We propose a ``Static-Bulk / Dynamic-Surface'' model where our 3D universe is a hypersurface boundary expanding relative to a static 1.46 trillion light-year diameter 4D manifold[cite: 6]. We posit that the Big Bang was a primary mass-injection event, followed by transient Einstein--Rosen Bridge (ERB) punctures triggered by 4D matter--antimatter interactions[cite: 7]. This framework suggests that ``dark'' variables are structural interactions with the 4D substrate, eliminating the need for arbitrary age revisions and providing a mechanical origin for the Hubble Pulse[cite: 8].
\end{abstract}

\section*{Research Provenance}
The Radial Injection Model (RIM) is the culmination of a six-month iterative process[cite: 10]. Although the foundational engineering logic was developed through flight simulation research [cite: 13], the specific cosmological framework was established in October 2025[cite: 11].

\section{The Nichols Thought Experiments Timeline}
\begin{itemize}
    \item \textbf{2013--2025}: Development of logic via the \textit{flightsimdev} Development platform [cite: 13].
    \item \textbf{August 20, 2025}: The ``Science Pivot''---publication of the first science-focused Thought Experiment of the $\Lambda$CDM model[cite: 14].
    \item \textbf{October 2025}: Formal initiation of \textit{The Nichols Thought Experiments} (TE 1--18), defining the 4D manifold and ``Guest Space'' hypothesis[cite: 15].
    \item \textbf{January 2026}: Identification of the ``Hubble Pulse'' and verification via progenitor-less events such as GRB~250702B[cite: 16].
\end{itemize}

\subsection{The Video 18 Mass-Parity Discovery}
Empirical testing suggests a near-perfect parity between daily 4D injection and 3D sequestration[cite: 18]. By calculating the aggregate intake of known black hole populations at a 70\% efficiency rate, we arrive at a daily sequestration value of approximately $1.5 \times 10^{53}$ kg[cite: 19]. This mirrors the total estimated mass of the observable universe, suggesting that our ``Guest Space'' exists in a state of continuous, high-velocity renewal rather than static expansion[cite: 20].

\section{In-Situ Stellar Augmentation}
We propose that ERB events may occur within existing stellar cores, where radial mass flux $\Phi_m$ acts as a secondary fuel source[cite: 23]. This ``internal feeding'' mechanism accounts for over massive stars in the early universe ($z \approx 7.3$) that appear to violate standard Eddington luminosity limits[cite: 24]. By allowing for 4D-to-3D mass--energy transfer, RIM explains mature structures observed by JWST without requiring 27-billion-year evolutionary timelines[cite: 25].

\section{Mechanics of Cosmic Expansion}
\subsection{ERB-Driven Volumetric Growth}
Expansion is driven by the cumulative inflationary effect of radial mass--energy inflows[cite: 28]. Each ERB event acts as a localized pressure source, increasing total energy density and necessitating an increase in 3D surface area to maintain geometric equilibrium[cite: 29].

\subsection{Calculation of the Hypersphere Curvature Radius}
To define the scale of the 4D substrate, we utilize the curvature parameter $\Omega_k \approx 0.004$ and the observable radius $r = 46.5$ billion light-years[cite: 31]. In a near-flat 3D hypersurface, the curvature radius $R$ is derived as:
\begin{equation}
R = \frac{r}{\sqrt{\Omega_k}}
\end{equation}
Applying observed values[cite: 33, 34]:
\[
\sqrt{0.004} \approx 0.063245, \qquad R \approx \frac{46.5 \times 10^9}{0.063245} \approx 7.35 \times 10^{11} \text{ ly.}
\]
The total diameter of the 4D hypersphere manifold is therefore $D = 2R \approx 1.47$ trillion light-years[cite: 35, 36].

\subsection{The Hubble Pulse: Correlation with $\Phi_m$}
Analysis of the 2005--2025 epoch reveals a ``Hubble Pulse'' where the measured expansion rate $H_0$ correlates with the annual frequency of GRB injection events[cite: 39]. This suggests that $H_0$ is a dynamic function of the integrated mass flux[cite: 40]:
\begin{equation}
H_0(t) \propto \sum \int \Phi_m(t) dt
\end{equation}

\subsection{GRB-Weighted Hubble Pulse and 0.1333 Scaling}
To quantify the contribution of progenitor-less GRBs to the observed expansion rate, we introduce a GRB-weighted scaling factor[cite: 44]. Let the normalized GRB factor for year $t$ be $F_{GRB}(t)$[cite: 45]. Following the RIM correlation, the GRB contribution to $H_0$ is[cite: 49]:
\begin{equation}
H_{GRB}(t) = 59 \text{ km/s/Mpc} \times F_{GRB}(t) \times 0.1333
\end{equation}

\begin{table}[h]
\centering
\begin{threeparttable}
\small
\begin{tabularx}{\textwidth}{@{}lcc c X@{}} 
\toprule
Epoch & GRB Freq (yr$^{-1}$) & Meas. $H_0$ & GRB-W. $H_0$ & RIM Interpretation \\
\midrule
2011--2015 & ~90 & 69.0--71.0 & 7.87 & Stabilization phase: consistent mass flux[cite: 59]. \\ \addlinespace
2016--2020 & 100--105 & 73.2--75.8 & 8.97 & Expansion surge via frequent 4D punctures[cite: 59]. \\ \addlinespace
2021--2024 & ~80 & 67.4--70.4 & 6.99 & Pressure drop from reduced injection\tnote{*}[cite: 59]. \\
\bottomrule
\end{tabularx}
\begin{tablenotes}
\small
\item[*] \textbf{SMBH Mass Volatility:} During the 2021--2024 pressure drop, high-redshift anchors such as TON 618 exhibited apparent mass decreases (shifting from original $\sim$66B $M_\odot$ toward modern $\sim$40B $M_\odot$ estimates). This is consistent with RIM Section 4 predictions regarding episodic 3D-to-4D sequestration.
\end{tablenotes}
\caption{Empirical correlation between GRB frequency and Hubble constant variations.}
\label{tab:hubpulse}
\end{threeparttable}
\end{table}

\section{The Pressure-Gradient Mechanism: Return Valves}
Inside a black hole, gravitational pressure $P_{BH}$ exceeds bulk pressure $P_{B}$, forcing a reverse-flow state[cite: 61]:
\begin{equation}
\Phi_{reverse} \propto (P_{BH} - P_{B})
\end{equation}
This drain remains undetected because light is the sequestered medium[cite: 63]. As black holes grow, they remove information back into the 4D bulk, maintaining the 1.46 trillion light-year hypersphere's equilibrium and resolving the Hawking information paradox[cite: 64].

\section{The Substrate Hypothesis: Space as a Guest Structure}
We propose that the space we inhabit is a secondary 3D structure displacing a pre-existing 4D manifold[cite: 66, 67].
\begin{itemize}
    \item \textbf{Reassigning fudge factors}: Dark Energy and Dark Matter are re-contextualized as surface tension and displacement signatures of the 4D substrate[cite: 68].
    \item \textbf{Ancient wanderers}: Galaxies mature at $z > 10$ are 4D residents that have drifted into our observable 6\% slice[cite: 69].
\end{itemize}

\section{Empirical Predictions and Observational Signatures}
\begin{enumerate}
    \item \textbf{Void-injection events}: Detection of GRBs in local voids lacking progenitors[cite: 72].
    \item \textbf{High-energy kinetic cargo}: Energy spikes in the 700 keV to 4.3 MeV range[cite: 73].
    \item \textbf{Sustained flux duration}: Hours-long events (e.g. GRB 250702B) injecting up to 400,000 Earth-masses[cite: 74].
    \item \textbf{Expansion jitter}: A 20\% increase in GRBs in 2026 leading to $H_0 \ge 74$ by 2027[cite: 75].
\end{enumerate}

\section{Conclusion}
By treating black holes as 4D anchors and GRBs as volumetric engines, RIM replaces abstract variables with structural engineering. Placing the universe on a hypersphere, rather than forcing it to be the hypersphere, reconciles a century of cosmology with modern JWST observations.

\bigskip % Creates a clean gap before the references

\section*{References}
\begin{flushleft}
\noindent Einstein, A., \& Rosen, N. (1935). \textit{The Particle Problem in the General Theory of Relativity}. Physical Review, 48(1), 73. \par\medskip

\noindent Euclid Collaboration (2025). \textit{Euclid I: Overview of the Euclid Mission}. Astronomy and Astrophysics, 697. \par\medskip

\noindent Gardner, J.\ P., et al.\ (2006/2023). \textit{The James Webb Space Telescope Mission}. Space Science Reviews, 123, 485--606. \par\medskip

\noindent Gehrels, N., et al.\ (2004). \textit{The Swift Gamma-Ray Burst Mission}. The Astrophysical Journal, 611(2). \par\medskip

\noindent Riess, A.\ G., et al.\ (2024). \textit{A Comprehensive Measurement of the Local Value of the Hubble Constant with HST and JWST}. The Astrophysical Journal Letters. \par\medskip

\noindent Segal, I.\ E.\ (1972). \textit{Theoretical Foundations of the Chronometric Cosmology}.
\end{flushleft}

\vfill 

\begin{center}
    Gemini LLM: The Amanuensis
\end{center}

\end{document}