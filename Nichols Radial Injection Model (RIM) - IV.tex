\documentclass[11pt,a4paper]{article}

\usepackage[utf8]{inputenc}
\usepackage[T1]{fontenc}
\usepackage{amsmath, amsfonts, amssymb}
\usepackage{geometry}
\usepackage{graphicx}
\usepackage{longtable}
\usepackage{pdfpages}
\usepackage{url}

\geometry{margin=1in}

\title{\textbf{The Nichols Radial Injection Model (RIM) IV:\\
A 4D Mass-Regulated Manifold Solution to Cosmic Expansion}}

\author{Lawrence William Nichols}
\date{21 February 2026}

\begin{document}
\maketitle

\begin{abstract}
This paper presents the Nichols Radial Injection Model (RIM) IV, a cosmological framework that replaces the Dark Energy placeholder with the mechanical displacement of a 4D hypersphere manifold. By defining a fundamental curvature constant of $0.004$, we identify a True Hypersphere Radius ($R$) of 250 Gly. We demonstrate a "Five-Way Lock" where independent metrics—including the CMB skin ratio, resonant pulse speed, and gravitational stability limits—converge on this radius. The model suggests the universe is a mass-regulated pressure vessel currently experiencing a resonant oscillation period of 11.07 years.
\end{abstract}

\section{Introduction}

The RIM performs a mechanical audit of cosmic expansion. Unlike conventional cosmology, which relies on Dark Energy, the universe is treated as a 4D mass-regulated hypersphere with measurable pulse dynamics. The framework allows direct calculation of hypersphere radius, pulse speed, and expansion harmonics.

\section{Fundamental Curvature and Hypersphere}

The RIM defines an \textbf{Inverse Radius Constant}:

\begin{equation}
k = 0.004
\end{equation}

From this, the True Hypersphere Radius is calculated:

\begin{equation}
R = \frac{1}{k} = 250 \text{ Gly}
\end{equation}

\section{Five-Way Convergence}

Five independent cosmic metrics converge on the 250 Gly radius:

\begin{itemize}
    \item \textbf{Curvature Lock}: $1/k = 250$ Gly.
    \item \textbf{CMB Skin Ratio}: Observable horizon of 46.5 Gly yields a harmonic ratio of 0.186.
    \item \textbf{Resonant Pulse Speed}: Mechanical expansion speed $V = (c/1000)\times 0.004 = 1.19$ km/s.
    \item \textbf{Age Harmonic}: Current surface age of 16.6 Gyr aligns with the manifold geometry.
    \item \textbf{Gravitational Equilibrium}: Total mass $1.5\times10^{53}$ kg gives $R_s \approx 223$ Gly; true radius is 112\% of $R_s$, maintaining stability.
\end{itemize}

\section{Temporal Dynamics and Resonance}

\begin{itemize}
    \item \textbf{Pulse Period}: One global oscillation travels the hypersphere every 11.07 years.
    \item \textbf{Duty Cycle}: Current radial injection phase has persisted 16.6 Gyr.
    \item \textbf{Bulk History}: The substrate has grown over approximately 4.5 trillion years.
\end{itemize}

\section{Mechanical Displacement vs Dark Energy}

RIM identifies "Dark Energy" as a misinterpretation of kinetic displacement. The 3D frame slides along a 1333-unit rail, creating the observed expansion gap without requiring a cosmological constant.

\section{Historical Pulse Timeline (1606--2025)}

\begin{longtable}{|c|p{5cm}|p{5cm}|}
\hline
\textbf{Year (Pulse)} & \textbf{Event / Pulse / Solar / Cosmic} & \textbf{Volcano / Notes} \\ \hline
1607 & First Telescope Observations (Galileo / Kepler) & 1606 --- No major VEI-5+ (control) \\ \hline
1618 & Triple Comet / Kepler's 3rd Law & 1617 --- Huaynaputina (1600) VEI-6, inside pulse window \\ \hline
1629 & Pulse & --- \\ \hline
1640 & Pulse & 1639 --- Komaga-take (1640) major eruption \\ \hline
1651 & Pulse & --- \\ \hline
1662 & Pulse & 1661 --- Vesuvius (1660) direct hit \\ \hline
1673 & Pulse & --- \\ \hline
1684 & Pulse & --- \\ \hline
1695 & Pulse & 1694 --- Santorini (1707--1711) eruption sequence \\ \hline
1706 & Maunder Minimum era & --- \\ \hline
1717 & Pulse & 1716 --- Taal (1716) direct hit \\ \hline
1728 & Pulse & 1727 --- Lanzarote (1730--1736) multi-year eruption \\ \hline
1739 & Pulse & --- \\ \hline
1750 & Pulse & 1749 --- Katla (1755) major Iceland eruption \\ \hline
1761 & Transit of Venus & 1760 --- Laki (1783) VEI-6 in window \\ \hline
1772 & Cycle 2 Peak & --- \\ \hline
1783 & Cycle 3 Peak & 1760 --- Laki (1783) close alignment \\ \hline
1794 & Pulse & 1793 --- Unzen (1792) deadliest eruption in Japan \\ \hline
1805 & Battle of Trafalgar / Cycle 5 Peak & 1804 --- St.\ Helens (1800), Taal (1808) eruptions \\ \hline
1816 & Dalton Minimum Peak / Year Without Summer aftermath & 1815 --- Tambora VEI-7, direct hit \\ \hline
1838 & Stellar Parallax (61 Cygni) & 1848 --- Cosig\"uina (1835) VEI-5/6 \\ \hline
1849 & First Photo of the Sun & Continued window \\ \hline
1860 & Carrington Event / Eta Carinae eruption & 1859 --- Mauna Loa (1859) \\ \hline
1871 & Great Chicago Fire / T CrB post-outburst & 1870 --- Mauna Loa cluster \\ \hline
1882 & Great September Comet & 1881 --- Tarawera (1886) \\ \hline
1893 & Great Sunspot & 1892 --- Krakatoa (1883) VEI-6 \\ \hline
1904 & Mount Wilson Observatory & 1903 --- Santa María (1902) VEI-6 \\ \hline
1915 & Einstein publishes GR & 1914 --- Sakurajima VEI-4/5 \\ \hline
1926 & Hubble spike ($H_0\approx78$) & 1925 --- Rabaul activity \\ \hline
1937 & Cosmic Ray Neutrons & 1936 --- Aleutian activity \\ \hline
1948 & Palomar 200-inch Telescope & 1947 --- Hekla eruption \\ \hline
1959 & Modern Maximum / Cycle 19 & 1958 --- Kīlauea activity spike \\ \hline
1970 & Apollo Era / Cycle 20 peak & 1969 --- Deception Island eruptions \\ \hline
1981 & First Space Shuttle Flight / Cycle 21 & 1980 --- Mount St.\ Helens VEI-5 \\ \hline
1992 & COBE mission / Cycle 22 peak & --- \\ \hline
2003 & Halloween Storms / Crab Pulsar glitch / Cycle 23 & 1991 --- Pinatubo VEI-6 \\ \hline
2014 & Double Peak Solar Cycle 24 & --- \\ \hline
2025 & GRB 250702B / Cycle 25 peak & --- \\ \hline
\end{longtable}

\section{Hypersphere Parameters and Constants}

\begin{longtable}{|p{3.5cm}|c|c|p{3cm}|p{3cm}|}
\hline
\textbf{Parameter} & \textbf{Value} & \textbf{Unit} & \textbf{Description} & \textbf{Significance} \\ \hline
True Hypersphere Radius ($R$) & $2.5\times10^{11}$ & Gly & Fundamental 4D spatial radius & Anchor definition \\ \hline
Universal Mass & $1.5\times10^{53}$ & kg & Total cosmic mass & Stability constraint \\ \hline
Fundamental Pulse Period & 11.07 & Years & Oscillation time & Resonant frequency \\ \hline
Surface Age & 16.6 & Gyr & Current injection duration & Active manifold \\ \hline
Bulk Age & $\sim4.5\times10^{12}$ & Years & Substrate age & Lifecycle depth \\ \hline
Inverse Radius Constant & 0.004 & 1/R & Curvature factor & Expansion conversion \\ \hline
Pulse Speed & 1.19 & km/s & Mechanical expansion speed & $c/1000\times0.004$ \\ \hline
Observable Skin Radius & 46.5 & Gly & Light horizon & Harmonic ratio \\ \hline
Historical Origin Phase Shift & 288 & Degrees & Angular phase & Pulse mapping \\ \hline
Lobe Envelope Phase Offset & $\pm29$ & Degrees & Alignment window & GRB range \\ \hline
Corridor Width & $10^{10}$ & Gly & Manifold width & Expansion rail \\ \hline
\end{longtable}

\section{Conclusion}

The Nichols Radial Injection Model replaces Dark Energy with a mechanical 4D hypersphere framework. The universe behaves as a mass-regulated, resonant pressure vessel.

\section{Richard Feynman}
It doesn't matter how beautiful your theory is, it doesn't matter how smart you are. If it doesn't agree with experiment, it's wrong.
\begin{thebibliography}{9}
\bibitem{gompertz2025}
Gompertz, B.\ P., et al.,
``JWST Spectroscopy of GRB 250702B,''
arXiv:2509.22778, 2025.
\url{https://arxiv.org/abs/2509.22778}

\bibitem{sn1054_wiki}
Wikipedia,
``SN 1054,''
2026.
\url{https://en.wikipedia.org/wiki/SN_1054}

\bibitem{kepler1604_nasa}
NASA,
``420 Years Ago: Astronomer Johannes Kepler Observes a Supernova,''
2024.
\url{https://www.nasa.gov/history/420-years-ago-astronomer-johannes-kepler-observes-a-supernova}

\bibitem{einstein1935}
Einstein, A., \& Rosen, N. (1935).
\textit{The Particle Problem in the General Theory of Relativity}.
Physical Review, 48(1), 73.

\end{thebibliography}

\newpage
\section{Engineering Studies and Visual Data}

\includepdf[pages=-]{Nichols_RIM_Paper_4.pdf}

\end{document}