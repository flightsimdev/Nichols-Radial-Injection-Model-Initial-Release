\documentclass[12pt]{article}

\usepackage[a4paper,margin=1.25in]{geometry}
\usepackage{amsmath,amssymb}
\usepackage{graphicx}
\usepackage{booktabs}
\usepackage{physics}
\usepackage{hyperref}
\usepackage{tabularx}
\usepackage{threeparttable}
\usepackage{setspace}
\usepackage{float} % Added for [H] table placement
\onehalfspacing
\usepackage{parskip}

\title{%
Nichols Radial Injection Model (RIM) and Radial ERB Inflows:\\
A Mechanism for Progenitor-less Astrophysical Events,\\
the Hubble Pulse, and 4D Substrate Interactions
}

\author{Lawrence William Nichols}

\date{January 21, 2026}

\begin{document}
\maketitle

\begin{abstract}
Current $\Lambda$CDM models struggle to account for the rapid formation of massive galaxies and hours-long Gamma-Ray Bursts (GRBs) observed by JWST. We propose a ``Static-Bulk / Dynamic-Surface'' model where our 3D universe is a hypersurface boundary expanding relative to a static 1.46 trillion light-year diameter 4D manifold. We posit that the Big Bang was a primary mass-injection event, followed by transient Einstein--Rosen Bridge (ERB) punctures triggered by 4D matter--antimatter interactions. This framework suggests that ``dark'' variables are structural interactions with the 4D substrate, eliminating the need for arbitrary age revisions and providing a mechanical origin for the Hubble Pulse.
\end{abstract}

\section*{Research Provenance}
The Radial Injection Model (RIM) is the culmination of a six-month iterative process. Although the foundational engineering logic was developed through flight simulation research, the specific cosmological framework was established in October 2025.
\newpage

\begin{center}\textbf{Revision: February 5, 2026 --- 29-Year Manifold Audit \& 200-Count Expansion Mapping Update}\end{center}

\begin{table}[h]
\centering
\begin{threeparttable}
\small
\begin{tabularx}{\textwidth}{@{}l c c c X@{}} 
\toprule
Epoch (7 Sets) & $F_{GRB}$ ($yr^{-1}$) & Formula $H_{calc}$ & Meas. $H_0$ & RIM Mechanical State \\
\midrule
2004--2006 & 90 & 70.7 & 73.0--73.5 & High Torque Injection Phase. \\ \addlinespace
2007--2009 & 86 & 67.5 & 67.0--68.5 & Transition to Respiration. \\ \addlinespace
2010--2013 & 95 & 74.6 & 72.0--74.0 & Peak Manifold Pressure. \\ \addlinespace
2014--2017 & 90 & 70.7 & 69.0--71.0 & Equilibrium Settling. \\ \addlinespace
2018--2021 & 103 & 81.0 & 73.2--75.8 & Secondary Surge (90 Spike). \\ \addlinespace
2022--2024 & 80 & 62.9 & 67.4--70.4 & 6\% Biennial Volumetric Loss. \\ \addlinespace
\midrule
\textbf{20-Year Mean} & \textbf{92.42} & \textbf{72.6} & \textbf{71.6} & \textbf{Equilibrium Set-Point} \\
\bottomrule
\end{tabularx}
\begin{tablenotes}
\small
\item[*] \textbf{Formula Validation:} $H_{calc} = 59 \times F_{GRB} \times 0.1333$. The $\approx 1.0$ delta between $H_{calc}$ (72.6) and Measured Mean (71.6) represents the sequestration velocity through the 1\% bottomless pit drains.
\end{tablenotes}
\caption{20-Year Audit: Correlation of 7 Data Epochs to the 71.6 Mean.}
\label{tab:finalaudit}
\end{threeparttable}
\end{table}

\section{The Nichols Thought Experiments Timeline}
\begin{itemize}
    \item \textbf{2013--2025}: Development of a mechanical/mathematical logic via the \textit{Nichols Thought Experiments} platform.
    \item \textbf{August 20, 2025}: The ``Science Pivot''---publication of the first science-focused video applying engineering principles to anomalies in the $\Lambda$CDM model.
    \item \textbf{October 2025}: Formal initiation of \textit{The Nichols Thought Experiments} (NTE 1--18), defining the 4D manifold and ``Guest Space'' hypothesis.
    \item \textbf{January 2026}: Identification of the ``Hubble Pulse'' and verification via progenitor-less events such as GRB~250702B.
\end{itemize}

\section{In-Situ Stellar Augmentation}
We propose that ERB events may occur within existing stellar cores, where radial mass flux $\Phi_m$ acts as a secondary fuel source. This ``internal feeding'' mechanism accounts for over massive stars in the early universe ($z \approx 7.3$) that appear to violate standard Eddington luminosity limits. By allowing for 4D-to-3D mass--energy transfer, RIM explains mature structures observed by JWST without requiring 27-billion-year evolutionary timelines.

\section{Structural Specification: The 4D+Time 3-Sphere Manifold}
The universe is modeled as a \textbf{3-Sphere} ($\mathbb{S}^3$) embedded within a 5D temporal bulk. This geometry creates a self-contained "Round Corridor" where the 5th dimension acts as the axis of recurrence.

\subsection{Metric and Torsion Dynamics}
The manifold is governed by a 5D metric where spatial curvature and temporal drift are coupled via the constant $\omega = 0.004$:
\begin{equation}
ds^2 = R^2(d\psi^2 + \sin^2\psi(d\theta^2 + \sin^2\theta d\phi^2)) + (dt + \omega \, dw)^2
\end{equation}

\subsection{Calibrated Dimensions}
\begin{itemize}
    \item \textbf{Transverse Width ($W$):} $10^{10} \text{ ly}$ (The diameter of the 3-sphere, ensuring local flatness).
    \item \textbf{Longitudinal Orbit ($L$):} $4.6 \text{ Tly}$ (The geodesic length of one temporal cycle).
    \item \textbf{Causal Horizon ($d_h$):} $46.5 \text{ Gly}$ (The limit of the 3D projection, or the "3.4-mile" equivalent).
    \item \textbf{Expansion Headroom ($\xi$):} $20$ (The volumetric density ratio of the 5D bulk).
\end{itemize}

\section{Mechanics of Cosmic Expansion}
\subsection{ERB-Driven Volumetric Growth}
Expansion is driven by the cumulative inflationary effect of radial mass--energy inflows. Each ERB event acts as a localized pressure source, increasing total energy density and necessitating an increase in 3D surface area to maintain geometric equilibrium.

\subsection{Calculation of the Hypersphere Curvature Radius}
To define the scale of the 4D substrate, we utilize the curvature parameter $\Omega_k \approx 0.004$ and the observable radius $r = 46.5$ billion light-years. In a near-flat 3D hypersurface, the curvature radius $R$ is derived as:
\begin{equation}
R = \frac{r}{\sqrt{\Omega_k}}
\end{equation}
Applying observed values:
\[
\sqrt{0.004} \approx 0.063245, \qquad R \approx \frac{46.5 \times 10^9}{0.063245} \approx 7.35 \times 10^{11} \text{ly.}
\]
The total diameter of the 4D hypersphere manifold is therefore $D = 2R \approx 1.47$ trillion light-years.

\subsection{The Hubble Pulse: Correlation with $\Phi_m$}
Analysis of the 2005--2025 epoch reveals a ``Hubble Pulse'' where the measured expansion rate $H_0$ correlates with the annual frequency of GRB injection events. This suggests that $H_0$ is a dynamic function of the integrated mass flux:
\begin{equation}
H_0(t) \propto \sum \int \Phi_m(t) dt
\end{equation}

\subsection{GRB-Weighted Hubble Pulse and 0.1333 Scaling}
To quantify the contribution of progenitor-less GRBs to the observed expansion rate, we introduce a GRB-weighted scaling factor. Let the normalized GRB factor for year $t$ be $F_{GRB}(t)$. Following the RIM correlation, the GRB contribution to $H_0$ is:
\begin{equation}
H_{GRB}(t) = 59 \text{ km/s/Mpc} \times F_{GRB}(t) \times 0.1333
\end{equation}

\section{Expansion Rate Mapping: The 0.78647 Scaling Table}
\begin{table}[h!]
\centering
\small
\begin{tabularx}{0.8\textwidth}{@{}X c X c@{}}
\toprule
GRB Count ($x$) & RIM Expansion ($y$) & GRB Count ($x$) & RIM Expansion ($y$) \\
\midrule
10  & 7.8647   & 110 & 86.5117  \\
20  & 15.7294  & 120 & 94.3764  \\
30  & 23.5941  & 130 & 102.2411 \\
40  & 31.4588  & 140 & 110.1058  \\
50  & 39.3235  & 150 & 117.9705  \\
60  & 47.1882  & 160 & 125.8352  \\
70  & 55.0529  & 170 & 133.6999  \\
80  & 62.9176  & 180 & 141.5646  \\
90  & 70.7823  & 190 & 149.4293  \\
100 & 78.6470  & 200 & 157.2940  \\
\bottomrule
\end{tabularx}
\caption{Linear Expansion Metrics for Manifold Load Management.}
\label{tab:lookup}
\end{table}
\begin{center}\textbf{Revision: February 11, 2026}\end{center}
\section{Comparative Galaxy Maturity: RIM vs. Standard Timeline}

To calculate the RIM Age, we use the formula:

\begin{equation}
\text{RIM Age} = \text{Std. Age} \times \frac{16.6\ \text{Gyr}}{13.8\ \text{Gyr}}
\label{eq:rimage}
\end{equation}

This scales the standard cosmological age (13.8 Gyr universe) to the 16.6 Gyr RIM timeline.

\begin{table}[H]
\centering
\small
\begin{tabularx}{\textwidth}{@{}l c c c X@{}}
\toprule
Astronomical Object & Redshift ($z$) & Std. Age (13.8 Gyr) & \textbf{RIM Age (16.6 Gyr)} & Status Result \\
\midrule
JADES-GS-z13-0 & 13.2 & ~320 Myr & \textbf{~2.1 Gyr} & Mature Galaxy \\
TON 618 & 2.21 & ~3.0 Gyr & \textbf{10.0 Gyr} & Natural Growth \\
Phoenix A & 1.47 & ~2.0 Gyr & \textbf{4.6 Gyr} & Natural Growth \\
CMB Horizon & 1100 & 380,000 yr & \textbf{380,000 yr} & Visibility Limit \\
\bottomrule
\end{tabularx}
\caption{Comparative Maturity: Resolution of the Growth Gap via 16.6 Gyr Timeline.}
\label{tab:galaxymaturity}
\end{table}

\section*{References}
\noindent Einstein, A., \& Rosen, N. (1935). \textit{The Particle Problem in the General Theory of Relativity}. Physical Review, 48(1), 73. \par\medskip
\noindent Euclid Collaboration (2025). \textit{Euclid I: Overview of the Euclid Mission}. Astronomy and Astrophysics, 697. \par\medskip
\noindent Gardner, J.\ P., et al.\ (2006/2023). \textit{The James Webb Space Telescope Mission}. Space Science Reviews, 123, 485--606. \par\medskip
\noindent Gehrels, N., et al.\ (2004). \textit{The Swift Gamma-Ray Burst Mission}. The Astrophysical Journal, 611(2). \par\medskip
\noindent Riess, A.\ G., et al.\ (2024). \textit{A Comprehensive Measurement of the Local Value of the Hubble Constant with HST and JWST}. The Astrophysical Journal Letters. \par\medskip
\noindent Segal, I.\ E.\ (1972). \textit{Theoretical Foundations of the Chronometric Cosmology}.

\end{document}
